\documentclass[8pt]{article}
\usepackage{extsizes}
\usepackage[utf8]{inputenc}
\usepackage[left=1cm,right=0.5cm,bottom=0.5cm,top=0.5cm]{geometry}
\usepackage{setspace}
\usepackage{graphicx}
\usepackage{amsmath}
\usepackage{mathabx}
\usepackage{enumitem}
\setstretch{0}
\setlist{nosep}

% 2 sides of paper max
% https://phys.libretexts.org/Courses/University_of_California_Davis/UCD%3A_Physics_9C__Electricity_and_Magnetism
\begin{document}
\textbf{Electrostatics}
\begin{enumerate}
    \item Gauss's law: For highly symmetric systems, you can use Gauss's law to find the electric field:
    \begin{enumerate}
        \item Use symmetry to determine the direction of $\vec{E}$ and what type of coordinate system it should be in.
        \item Pick a surface that $\vec{E}$ is entirely perpendicular to (and where the magnitude of the field is constant)
        \item Calculate the flux of the electric field relative to the bounding shape you used.
        \item Apply Gauss's law to find the magnitude of the field.
        \item Ex: for an infinite sheet you can choose a cylinder to calculate the flux since it spreads infinitely in all directions.
        \item Relevant formulas: $\int_S \vec{E} \cdot d\vec{A} = \Phi = \frac{Q_{en}}{\epsilon_0}$
    \end{enumerate}
\end{enumerate}
\textbf{Moving electric forces}
\begin{enumerate}
    \item KCL and KVL: Net voltage drop around a closed loop is 0; Net current into a node = net current out of the node.
\end{enumerate}
\textbf{Magnetism}
\begin{enumerate}
    \item When calculating $\vec{F}_{mag}$, note that $\vec{l}$ is parallel to the direction of current flow.
    \item $\vec{B}_{circle} = \frac{\mu_0 I}{2R}$
    \item Biot-Savart: In the $\vec{B}_{line}$ integral, $R$ represents distance between point and the closest point that emits ${\vec{B}}$.
    \item In a uniform magnetic field, a semicircle experiences no net $\vec{F}_{mag}$; it cancels.
    \item To integrate a circle of wire, use $\vec{r} = R \hat{r}$ and $d\vec{l} = R \phi d\hat{\phi}$.
    \item Magnetic force RHR: Fingertips are current, palm direction is magnetic field, thumb is magnetic force.
    \item Ampere's law: For highly symmetric systems, you can use Ampere's law to find the magnetic field:
    \begin{enumerate}
        \item Use symmetry to determine the general direction of $\vec{B}$
        \item Choose a closed path \textit{around} parallel to $\vec{B}$ where the magnitude of the field is constant.
        \item Make sure you're using the right integrand (is it radial?) and $d\vec{A}$ is normal to the plane of the closed path.
        \item Calculate the enclosed current and use Ampere's law to solve for $\vec{B}$.
        \item Relevant formulas: $\int_C \vec{B} \cdot d\vec{l} = \mu_0 I_{en}$, $I_{en} = \int_S \vec{J} \cdot d\vec{A}$
    \end{enumerate}
    \item $\vec{\tau}$ = Magnetic torque, $\vec{\mu}$ = Magnetic dipole moment.
\end{enumerate}
\textbf{Electromagnetism}
\begin{enumerate}
    \item The area vector $\vec{A}$ is perpendicular to the plane where the object exists.
    \item Lenz's Law right-hand rule: if $\frac{d \Phi}{dt} < 0$, $\vec{B}_{ind} \upuparrows \vec{B}_{ext}$. if $\frac{d \Phi}{dt} > 0$, $\vec{B}_{ext} \updownarrows \vec{B}_{ext}$.
    \item You can derive the wave equation from the source-free versions of Maxwell's equations:
    \begin{enumerate}
        \item Take the curl ($\nabla \times$) of the relation between $\vec{E}$ and $-\frac{d\vec{B}}{dt}$
        \item $-\nabla^2 \vec{E} = -\mu_0 \epsilon_0 \frac{\partial^2 \vec{E}}{\partial t}$
    \end{enumerate}
\end{enumerate}
\textbf{Units}
\begin{itemize}
    \item $[P] (power) = W = \frac{kg \cdot m^2}{s^3}$
    \item $[F] (force) = N = \frac{kg \cdot m}{s^2}$
    \item $[U] (energy) = J = \frac{kg \cdot m^2}{s^2}$
\end{itemize}
\textbf{Misc}
\begin{itemize}
    \item $\hat{r} = \cos{\theta} \hat{i} + \sin{\theta} \hat{j}$
    \item $\hat{\phi} = -\sin{\theta} \hat{i} + \cos{\theta} \hat{j}$
\end{itemize}
\end{document}
