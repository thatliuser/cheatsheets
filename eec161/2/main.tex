\documentclass[8pt]{article}
\usepackage{extsizes}
\usepackage[utf8]{inputenc}
\usepackage[left=1cm,right=0.5cm,bottom=0.5cm,top=0.5cm]{geometry}
\usepackage{setspace}
\usepackage{graphicx}
\usepackage{amsmath}
\usepackage{enumitem}
\setstretch{0}
\setlist{nosep}

% 2 sides of paper max
% Watch the lecture on Chapter 13, he probably goes over what's covered
\begin{document}
\textbf{Set Theory}
\begin{itemize}
    \item $S$ is the universal set - the set that contains all things of interest.
    \item Two sets $A$ and $B$ are mutually exclusive if $A \cap B = \emptyset$.
    \item Two sets $A$ and $B$ are collectively exhaustive if $A \cup B = S$.
    \item Outcome - singular possible observation of an experiment
    \item Sample space - all possible outcomes of an experiment ($S$)
    \item Event - collection of outcomes
\end{itemize}
\textbf{Probability Axioms}
\begin{itemize}
    \item $P[S] = 1$
    \item $P[\emptyset] = 0$
    \item For any event $A$, $P[A^c] = 1 - P[A]$.
    \item For any event $A$, $P[A] \geq 0$.
    \item For mutually exclusive events $A$ and $B$, $P[A \cup B] = P[A] + P[B]$.
    \item For any events $A$ and $B$, $P[A \cup B] = P[A] + P[B] - P[A \cap B]$.
    \item If $A \subset B$, then $P[A] \leq P[B]$.
    \item The probability of an event is the sum of the probabilities of the outcomes.
\end{itemize}
\textbf{Conditional Probability}
\begin{itemize}
    \item Conditional probability of an event $A$ occurring given that event $B$ occurred:
    \item $P[A | B] = \frac{P[AB]}{P[B]}$.
    \item A set can be split up and re-added together, and so can its probabilities:
    \item $A = A \cap B + A \cap C$ where $B \cap C = \emptyset$ and $B \cup C = S$
    \item $P[B | A] = \frac{P[A|B]P[B]}{P[A]}$.
    \item Two events $A$ and $B$ are independent if $P[AB] = P[A]P[B]$.
    \item If two events $A$ and $B$ are independent, $A^c$ and $B^c$ are also independent.
\end{itemize}
\textbf{Sequential Experiments}
\begin{itemize}
    \item Permutations: $(n)_k = \frac{n!}{(n - k)!}$, combinations: ${n \choose k} = \frac{n!}{k!(n - k)!}$
    \item Given $m$ objects, there are $m^n$ ways to choose an ordered sequence of $n$ objects.
    \item For $n$ subexperiments ($S = \{ 0, 1 \}$), the number of sequences with 0 appearing $n_0$ and 1 appearing $n_1 = n - n_0$ times is ${n \choose n_1}$.
\end{itemize}
\textbf{Discrete Random Variables}
\begin{itemize}
    \item PMF (Probability mass function) | $P_X(x) = P[X = x]$
    \item CDF (Cumulative distribution function) | $F_X(x) = P[X \leq x]$
    \item For CDF, only the non-zero (infinite) slopes are discrete cases.
    \item $F_X(b) - F_X(a) = P[a < X \leq b]$
    \item Expected value | $E[X] = \mu_X = \sum_{x \in S_X} xP_X(x)$
    \item Derived random vars | $Y = g(X)$
    \item $P_Y(y) = \sum_{x:g(x) = y} P_X(x)$
    \item $E[Y] = \mu_Y = \sum_{x \in S_X} g(x) P_X(x)$
    \item $E[X - \mu_X] = 0$, $E[aX + b] = aE[X] + b$
    \item $Var[X] = E[X^2] - \mu_X^2$, $n$th moment is $E[X^n]$, $n$th central moment is $E[(X - \mu_X)^n]$.
    \item $Var[aX + b] = a^2 Var[X]$
    \item Standard deviation | $\sigma_X = \sqrt{Var[X]}$
    \item
    \begin{tabular}{|c|c|c|c|}
        \hline
        Name & PMF & $E[X]$ & $Var[X]$ \\
        \hline
        Bernoulli & $P_X(x) = \begin{cases}
            1 - p & x = 0, \\
            p & x = 1, \\
            0 & else \\
        \end{cases}$ & $p$ & $p(1 - p)$ \\
        Geometric & $P_X(x) = \begin{cases}
            p (1 - p)^{x - 1} & x = 1, 2, \ldots, \\
            0 & else \\
        \end{cases}$ & $1/p$ & $(1 - p)/p^2$ \\
        Binomial & $P_X(x) = \begin{cases}
            {n \choose x} p^x (1 - p)^{n - x} & n \geq 1 \\
        \end{cases}$ & $np$ & $np(1 - p)$ \\
        Pascal & $P_X(x) = \begin{cases}
            {x - 1 \choose k - 1} p^k (1 - p)^{x - k} & k \geq 1 \\
        \end{cases}$ & $k/p$ & $k(1 - p)/p^2$ \\
        Poisson & $P_X(x) = \begin{cases}
            \alpha^x e^{-\alpha}/x! & x = 0, 1, \ldots, \\
            0 & else \\
        \end{cases}$ & $\alpha$ & $\alpha$ \\
        Discrete Uniform & $P_X(x) = \begin{cases}
            1/(l - k + 1) & x  = k, k + 1, \ldots, l \\
            0 & else \\
        \end{cases}$ & $(k + l)/2$ & $(l - k)(l - k + 2)/12$ \\
        \hline
    \end{tabular}
\end{itemize}
\textbf{Continous Random Variables}
\begin{itemize}
    \item PDF (Probability distribution function) | $f_X(x) = \frac{dF_X(x)}{dx}$
    \item PDFs \textbf{never} have a negative slope. Some sort of line from 0 to 1.
    \item CDF (Cumulative distribution function) from PDF | $F_X(x) = \int_{-\infty}^x f_X(u)du$
    \item $\int_{-\infty}^\infty f_X(x)dx = 1$ (integral of CDF over all X must equal 1)
    \item $P[x_1 < X \leq x_2] = \int_{x_1}^{x_2} f_X(x)dx$
    \item $E[X] = \int_{-\infty}^{\infty} x f_X(x)dx$
    \item $E[g(X)] = \int_{-\infty}^{\infty} g(x) f_X(x)dx$
    \item
    \begin{tabular}{|c|c|c|c|c|}
        \hline
        Name & PDF & CDF & $E[X]$ & $Var[X]$ \\
        \hline
        Uniform & $f_X(x) = \begin{cases}
            1/(b - a) & a \leq x < b, \\
            0 & else \\
        \end{cases}$ & $F_X(x) = \begin{cases}
            0 & x \leq a, \\
            (x - a)/(b - a) & a < x \leq b, \\
            1 & x > b \\
        \end{cases}$ & $(b + a)/2$ & $(b - a)^2/12$ \\
        Exponential & $f_X(x) = \begin{cases}
            \lambda e^{-\lambda x} & x \geq 0, \\
            0 & else \\
        \end{cases}$ & $F_X(x) = \begin{cases}
            1 - e^{-\lambda x} &x \geq 0, \\
            0 & else \\
        \end{cases}$ & $1/\lambda$ & $1/\lambda^2$ \\
        Erlang & $f_X(x) = \begin{cases}
            \lambda^n x^{n - 1} e^{-\lambda x}/(n - 1)! & x \geq 0, \\
            0 & else \\
        \end{cases}$ & & $n/\lambda$ & $n/\lambda^2$ \\
        Gaussian & $f_X(x) = \frac{1}{\sqrt{2 \pi \sigma^2}} e^{-(x - \mu)^2/2 \sigma^2}$
        & & $\mu$ & $\sigma^2$ \\
        \hline
    \end{tabular}
\end{itemize}
\textbf{Gaussian}
\begin{itemize}
    \item Standard normal func: $\Phi(z) = \frac{1}{\sqrt{2 \pi}} \int_{-\infty}^z e^{-u^2/2} du$
    \item $F_X(x) = \Phi\left(\frac{x - \mu}{\sigma}\right)$
    \item $P[a < X \leq b] = \Phi\left(\frac{b - \mu}{\sigma}\right) - \Phi\left(\frac{a - \mu}{\sigma}\right)$
    \item $\Phi(-z) = 1 - \Phi(z)$, $Q(z) = 1 - \Phi(z)$
\end{itemize}
\textbf{Multiple Random Variables}
\begin{itemize}
    \item Joint CDF | $F_{X,Y}(x,y) = P[X \leq x, Y \leq y]$
    \item Properties | $0 \leq F_{X,Y}(x, y) \leq 1$ | $F_{X,Y}(\infty, \infty) = 1$ | $F_X(x) = F_{X,Y}(x, \infty)$, same for y | $F_{X,Y}(x, \infty) = 0$, same for y
    \item Joint PMF | $P_{X,Y}(x, y) = P[X = x, y = y]$, Marginal PMF | $P_X(x) = \sum_{y \in S_Y} P_{X,Y}(x, y)$, flipped for y
    \item Joint PDF | $f_{X,Y}(x, y) = \frac{\partial^2 F_{X,Y}(x, y)}{\partial x \partial y}$ and $F_{X,Y}(x, y) = \int_{-\infty}^x \int_{-\infty}^y f_{X,Y}(u, v) dv du$
    \item Properties of Joint PDF | $f_{X,Y}(x, y) \geq 0$ always, and $\int_{-\infty}^\infty \int_{-\infty}^\infty f_{X,Y}(x, y) dxdy = 1$
    \item Marginal PDF | $f_X(x) = \int_{-\infty}^\infty f_{X,Y}(x, y)dy$, $f_Y(y) = \int_{-\infty}^\infty f_{X,Y}(x, y)dx$
    \item Independence | X and Y are independent only if $P_{X,Y}(x, y) = P_X(x) P_Y(y)$ for discrete, and $f_{X,Y}(x, y) = f_X(x)f_Y(y)$ for continuous.
    \item Expected value of $W = g(X, Y)$ | $E[W]= \sum_{x \in S_X} \sum_{y \in S_Y} g(x, y) P_{X,Y}(x, y)$ for discrete, $E[W] = \int_{-\infty}^\infty \int_{-\infty}^\infty g(x, y)f_{X,Y}(x, y)dxdy$ for continuous.
    % Might want to condense this line later
    \item $E[X + Y] = E[X] + E[Y]$
    \item Correlation $r_{X,Y} = E[XY]$ | $Cov[X, Y] = r_{X,Y} - \mu_X \mu_Y$ | $Var[X + Y] = Var[X] + Var[Y] + 2Cov[X, Y]$
    \item Correlation coefficient $\rho_{X,Y} = \frac{Cov[X, Y]}{\sigma_X \sigma_Y}$ | $-1 \leq \rho_{X,Y} \leq 1$
    \item If $X = Y$, $Cov[X, Y] = Var[X]$, $r_{X,Y} = E[X^2]$.
    \item If $r_{X,Y} = 0$, $X$ and $Y$ are orthogonal. If $Cov[X, Y] = 0$, $X$ and $Y$ are uncorrelated.
    \item For independent vars $X$ and $Y$, $r_{X,Y} = E[XY] = E[X]E[Y]$, $Cov[X, Y] = \rho_{X,Y} = 0$, $Var[X + Y] = Var[X] + Var[Y]$.
    % TODO: Bivariate Gaussian distributions? Not sure how useful these are
    % TODO: Do multivariate probability models? Maybe not, chapter 8 seems to cover matrix ops
\end{itemize}
\textbf{Models of Derived Random Variables}
\begin{itemize}
    \item Discrete | $P_W(w) = \Sigma_{(x, y): g(x,y) = w} P_{X,Y}(x, y)$
    \item Continuous, one variable | To find $F_W(w)$, find $W = g(X)$ and solve in terms of $X$, then plug into $F_X(x)$.
    \item Continuous, two variables | To find $F_W(w)$, find $W = g(X, Y)$ and use MAT 21D methods to find the bounds of integration.
    \item $F_W(w) = P[W \leq w] = \iint_{g(x,y) \leq w} f_{X,Y}(x, y) dx dy$
    \item Ex. if $W = max(X, Y)$, $F_W(w) = F_{X,Y}(w, w) = \int_{-\infty}^w \int_{-\infty}^w f_{X,Y}(x, y) dx dy$
    \item For both cases, to find $f_W(w)$, take the derivative of $F_W(w)$.
    \item For both cases, be careful of jump discontinuities in $F_W(w)$. If these exist, use $\delta(t)$ times the jump value to compensate.
    \item Sum of two random variables: $W = X + Y$ | PDF $f_W(w) = \int_{-\infty}^\infty f_{X,Y}(x, w - x) = \int_{-\infty}^\infty f_{X,Y}(w - y, y) dy$
\end{itemize}
\textbf{Conditional Models}
\begin{itemize}
    \item Discrete conditional PMF | $P_{X|B}(x) = \begin{cases}
        \frac{P_X(x)}{P[B]} & x \in B, \\
        0 & otherwise \\
    \end{cases}$
    \item Continuous conditional CDF | $f_{X|B}(x) = \begin{cases}
        \frac{f_X(x)}{P[B]} & x \in B, \\
        0 & otherwise \\
    \end{cases}$
    \item Discrete conditional expected val | $E[X|B] = \Sigma_{x \in B} xP_{X|B}(x)$
    \item Continous conditional expected val | $E[X|B] = \int_{-\infty}^\infty xf_{X|B}(x)dx$
    \item Conditonal variance | $Var[X|B] = E[X^2|B] - E[X|B]^2$
    \item Continuous condtional CDF of another var | $f{X|Y}(x|y) = \frac{f_{X,Y}(x, y)}{f_Y(y)}$
    \item $E[E[g(X)|Y]] = E[g(X)]$
\end{itemize}
\textbf{Vector Notation}
\begin{itemize}
    % TODO: IDKDLSFJDKlj LEHJXDXDXDXD
    \item
\end{itemize}
\textbf{Sums of Random Variables}
\begin{itemize}
    \item Given $W_n = \Sigma_{i = 1}^n X_i$,
    \item $E[W_n] = \Sigma_{i = 1}^n E[X_i]$.
    \item $Var[W_n] = \Sigma_{i = 1}^n Var[X_i] + 2 \Sigma_{i = 1}^{n - 1} \Sigma_{j = i + 1}^n Cov[X_i, X_j]$.
    \item MGFs (Moment generating functions) | $\Phi_X(s) = E[e^{sX}]$
    \item
    \begin{tabular}{|c|c|}
        \hline
        Name & $\Phi_X(s)$ \\
        \hline
        Bernoulli & $1 - p + pe^s$ \\
        Geometric & $\frac{pe^s}{1 - (1 - p)e^s}$ \\
        Binomial & $(1 - p + pe^s)^n$  \\
        Pascal & $(\frac{pe^s}{1 - (1 - p)e^s})^k$ \\
        Poisson & $e^{\alpha(e^s - 1)}$ \\
        Discrete Uniform & $\frac{e^{sk} - e^{s(l + 1)}}{1 - e^s}$ \\
        Uniform & $\frac{e^{bs} - e^{as}}{s(b - a)}$ \\
        Exponential & $\frac{\lambda}{\lambda - s}$ \\
        Erlang & $\left(\frac{\lambda}{\lambda - s}\right)^n$ \\
        Gaussian & $e^{s\mu + s^2\sigma^2/2}$ \\
        \hline
    \end{tabular}
    \item $\Phi_W(s) = \Sigma_{i = 1}^n \Phi_{X_i}(s)$
    \item Central limit theorem | Given $n$ independent random vars, the distribution is approximately Gaussian.
    \item $F_{W_n}(w) \approx \Phi\left(\frac{w - n \mu_X}{\sqrt{n \sigma_X^2}}\right)$
\end{itemize}
\end{document}
