\documentclass{article}
\usepackage[utf8]{inputenc}
\usepackage[left=1cm,right=0.5cm,bottom=0.5cm,top=0.5cm]{geometry}
\usepackage{setspace}
\usepackage{graphicx}
\usepackage{amsfonts}
\setstretch{0}

\begin{document}
\section{Two source interference}
\vspace{-3mm}
\textbf{Constructive} | $d\sin{(\theta)} = m\lambda : m \in \mathbb{Z}$; \textbf{Destructive} | $d\sin{(\theta)} = (m + \frac{1}{2})\lambda$, $d$ = slit distance, $\lambda$ = wavelength. \\
\textbf{Distance to $m^{th}$ band} | $y_m = R\tan{(\theta_m)}$; $y_m = R\frac{m\lambda}{d}$ for ``small angles" only, $R$ = slit $\rightarrow$ screen distance. \\
\textbf{Phase difference} | $\phi = \frac{2\pi}{\lambda}(r_2 - r_1)$, $r_2 -r_1$ = path distance difference. \\
\textbf{Electic field amplitude} | $E_P = 2E\left|\cos{\frac{\phi}{2}}\right|$, $E$ = amplitude from one src \\
\textbf{Refractive indices} | $\lambda = \frac{\lambda_0}{n}$, $\lambda_0$ = wavelength in vacuum, $n$ = refractive idx \\
\vspace{-1mm}
\textbf{Intensity} | $I = I_{max}\cos^2{\frac{\phi}{2}}$
\vspace{-5mm}
\section{Thin film interference}
t = film thickness \\
\textbf{$n_a > n_b$} | \textbf{construct} | $2t = m\lambda$ | \textbf{destruct} | $2t = (m + \frac{1}{2})\lambda$ \\
\textbf{$n_a < n_b$} | \textbf{construct} | $2t = (m + \frac{1}{2})\lambda$ | \textbf{destruct} | $2t = m\lambda$
\vspace{-5mm}
\section{Single slit diffraction}
\vspace{-3mm}
\textbf{Dark fringe} | $\sin\theta = \frac{m\lambda}{a}$, $\theta$ = angle to fringe \\
\textbf{Intensity} | $I = I_0\left\{\frac{\sin{(\beta/2)}}{\beta/2}\right\}^2$, $\beta = \frac{2\pi}{\lambda}a\sin{\theta}$, $a$ = slit width, USE RADIANS!
\vspace{-5mm}
\section{Multi slit diffraction}
\vspace{-2mm}
\textbf{Minima} | With $n$ slits, there are $n - 1$ evenly spaced minima for each maxima. \\
\textbf{Diffraction grating (maxima)} | $d\sin\theta = m\lambda$ \\
\textbf{X-ray diffraction maxima (Bragg condition)} | $2d\sin\theta = m\lambda$, $d$ = distance between adj. rows of atoms \\
\textbf{Circular apertures} | \textbf{bright} | $\sin\theta_1 = 1.22\frac{\lambda}{D}$, $\sin\theta_2 = 2.23\frac{\lambda}{D}$, $\sin\theta_3 = 3.24\frac{\lambda}{D}$ \\
\textbf{dark} | $\sin\theta_1 = 1.63\frac{\lambda}{D}$, $\sin\theta_2 = 2.68\frac{\lambda}{D}$, $\sin\theta_3 = 3.70\frac{\lambda}{D}$, $D$ = aperture diameter
\vspace{-5mm}
\section{Thermal expansion}
\vspace{-2mm}
\textbf{Linear} | $\Delta L = \alpha L_0 \Delta T$, $\alpha$ = coefficient of material, $L_0$ = orig. length, $\Delta T$ = temp. change \\
\textbf{Volume} | $\Delta V = \beta V_0 \Delta T$, $\beta = 3\alpha$, another coefficient \\
\textbf{Young's modulus} $\gamma = \frac{F/A}{\Delta L / L_0}$ \\
\textbf{Stress} | $\frac{F}{A} = -\gamma \alpha \Delta T$, $L$ = length, $F$ = force 
\vspace{-5mm}
\section{Heat}
\vspace{-2mm}
ALWAYS USE KELVIN!!! \\
\textbf{Units} | 1 cal = 4.186 joules, 1 Btu = 778 ft $\cdot$ lb = 252 cal = 1055 joules \\
\textbf{Specific heat} | $c = \frac{1}{m}\frac{dQ}{dT}$ \\
\textbf{Temperature change (kg)} | $Q = mc \Delta T$, Q = heat emitted, m = mass (kg), $\Delta T$ = temp. change, c ($\frac{J}{kg \cdot K}$) \\
\textbf{Temperature change (moles)} | $Q = mC \Delta T$, m = mass (moles), C ($\frac{J}{mol \cdot K}$) \\
\textbf{kg to mol} | $m = nM$, m (kg), n (mol), M ($\frac{kg}{mol}$) \\
\textbf{Phase change} | $Q = \pm mL$, L = latent heat (fusion, vaporization) \\
\vspace{-5mm}
\section{Heat transfer}
\vspace{-2mm}
\textbf{Heat current in conduction} | $H = \frac{dQ}{dt} = kA\frac{T_H - T_C}{L}$, k = rod thermal conductivity, A = cross sectional area, $T_H$ = hot, $T_C$ = cold, $L$ = length \\
\textbf{Thermal resistance} | $R = \frac{L}{k}$ \\
\textbf{Radiation} | $H = Ae\sigma T^4$, $H$ = heat current in radiation, $\sigma$ = boltzmann constant, $A$ = area of emitting surface \\
\textbf{Net radiation} | $H_{net} = Ae\sigma(T^4 - T_s^4)$, $T_s$ = surrounding temp \\
\vspace{-5mm}
\section{Thermal properties of matter}
\vspace{-2mm}
\textbf{Ideal gas equation} | $pV = nRT = Nk_BT$, p = pressure, v = vol, n = moles, $R = 8.314 \frac{J}{mol \cdot K} = 0.08206 \frac{L \cdot atm}{mol \cdot K}$, T = temp, $k_B = 1.38 \cdot 10^{-23} \frac{J}{kg}$ \\
\textbf{Density} | $\rho = \frac{pM}{RT}$ \\
\textbf{Avogadro's number} | $N_A = 6.02214076 \cdot 10^{23} \frac{molecules}{mol}$ \\
\textbf{Kinetic energy} | $K_{tr} = \frac{3}{2}nRT$ (J, amount of ideal gas), $\frac{1}{2}m(v^2)_{av} = \frac{3}{2}kT$ (amount per molecule of gas) \\
\textbf{Root mean square speed} | $v_{rms} = \sqrt{(v^2)_{av}} = \sqrt{\frac{3kT}{m}} = \sqrt{\frac{3RT}{M}}$, m = kg, M = molar mass \\
\textbf{Mean free path} | $\lambda = vt_{mean} = \frac{V}{4\pi\sqrt{2}r^2N}$, $t_{mean}$ = avg time between collisions, r = molecule radius, N = \# molecules
\vspace{-5mm}
\section{Heat capacity}
\vspace{-2mm}
\textbf{Monatomic ideal gas particles} | $C_V = \frac{3}{2}R$ \\
\textbf{Diatomic ideal gas particles} | $C_V = \frac{5}{2}R$ \\
\textbf{Triatomic ideal gas particles} | $C_V = \frac{7}{2}R$ \\
\textbf{Solids} | $C_V = 3R$ \\
\textbf{Maxwell-boltzmann distribution} | $f(v) = 4\pi\left(\frac{m}{2\pi kT}\right)^{3/2}v^2e^{-mv^2/2kT}$, m = gas molecule weight
\pagebreak

\end{document}
